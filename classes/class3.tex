\section{Class 3 - 1/03/21}
During this class we will talk about the passage from the time domain to frequency. We do that because in frequency domain we do not need anymore to do the derivate in time, and thus the calculations simplify a lot. Another reason is that is more important to look at the signal on the corner frequency other than all over the spectrum
\subsection*{EMF in phasor domain}
Now we take a look at the electrical field propagating on the $z$ axe: $\overline{E}(z,t)=E_x\,\jhat+E_y\,\ihat+E_z\,\khat$, and we explore all the element:
\begin{align}\label{eq:electric_field_3_axes_phase}
    \begin{split}
        E_x(z,t)&=E_{x_0}\,\cos(\omega\,t-\beta\,z+\varphi_x)\\[5pt]
        E_y(z,t)&=E_{y_0}\,\cos(\omega\,t-\beta\,z+\varphi_y)\\[5pt]
        E_z(z,t)&=E_{z_0}\,\cos(\omega\,t-\beta\,z+\varphi_z)
    \end{split}
\end{align}
Note that here $\omega$ and $\beta$ does not change, but $\varphi$ does, this is not very important, but it is just a note.\\
From \cref{eq:electric_field_3_axes_phase}, and exploiting the cos property:
\begin{equation*}
    \cos(\alpha + \beta )=\cos\alpha\, \cos\beta - \sin\alpha \, \sin\beta
\end{equation*}
By considering $\alpha = \omega\, t$ and $\beta = \beta z+\varphi+z$ we obtain the total equation of the EMF on $z$ direction:
\begin{align}
    \begin{split}
        &\overline{E}(z,t)=E_x\,\jhat+E_y\,\ihat+E_z\,\khat=\\[5pt]
        &=\cos(\omega\,t)[E_{x_0}\cos(\varphi-\beta\,z)\ihat+E_{y_0}\cos(\varphi_y-\beta\,z)\jhat+E_{z_0}\cos(\varphi_z-\beta\,z)\khat]+\\[5pt]
        &-\sin(\omega\,t)[E_{x_0}\sin(\varphi-\beta\,z)\ihat+E_{y_0}\sin(\varphi_y-\beta\,z)\jhat+E_{z_0}\sin(\varphi_z-\beta\,z)\khat])=\\[5pt]
        &=\overline{E}_1\,\cos(\omega\,t)+\overline{E}_2\,\sin(\omega\,t)
    \end{split}
\end{align}
First of all we notice that $\overline{E}_1$ and $\overline{E}_2$ are vectors that are only in function of space (and that is great and useful).\\
Then we can write:
\begin{equation}
    \overline{E}(z,t)=\operatorname{Re}\left\{(\overline{E}_1+j\overline{E}_2)[\cos(\omega t)+j \sin(\omega t)]\right\}=\operatorname{Re}\left\{(\overline{E}_1+j\overline{E}_2)e^{j\omega t}\right\}
\end{equation}
With this notation we can introduce the phasor 
\begin{equation}
\phas{E}(z)=\overline{E}_1+j\overline{E}_2
\end{equation}
And we will use this notation to describe the EMF in complex notation.
\begin{equation}\label{eq:electric_field_in_time}
    \overline{E}(z,t)=\operatorname{Re}\left\{\phas{E}(z)e^{j\omega t}\right\}
\end{equation}
If we want to go from time domain to phasor, we need to find $E_1$ and $E_2$. To do the opposite we need to use \cref{eq:electric_field_in_time}.\\
Now let's give a look at the derivative of the field:
\begin{equation}
    \frac{\partial\overline{E}(z,t)}{\partial t}=\operatorname{Re}\left\{\frac{\partial}{\partial t}\phas{E}e^{j\omega t}\right\}=\operatorname{Re}\left\{j\omega\,\phas{E}e^{j\omega t}\right\}
\end{equation}
With this trick we can use the derivative by only multiplying $j\omega$.\\
That being said, we can give the definition of electric or magnetic field that propagates in a general direction:
\begin{align}
    \begin{split}
        &\overline{E}(\overline{r},t)=\operatorname{Re}\left\{\phas{E}(\overline{r})e^{j\omega t}\right\}\\[5pt]
        &\overline{H}(\overline{r},t)=\operatorname{Re}\left\{\phas{H}(\overline{r})e^{j\omega t}\right\}\\[5pt]
    \end{split}
\end{align}
Now in phasor domain, we can write the maxwell equations that we have already seen in \cref{eq:maxwell_system_2}, but with a simpler notation.
\begin{align}\label{eq:maxwell_equation_in_phasor}
    \begin{split}
        &\nabla\times\phas{E}=-j\omega \,\phas{B}\\[5pt]
        &\nabla\times\phas{H}=j\omega \,\phas{D}+\phas{J}_\sigma+\phas{J}_i\\[5pt]
    \end{split}
\end{align}
Obviously this is true for any general direction $\overline{r}$, evn if we didn't mentioned that for a better notation elegance.\\
Another important thing that we need to stress is the relation of the other vectors that can be represented in the phasor space:
\begin{center}
    \begin{tabular}{ c c c }
        $\phas{D}=-\varepsilon \phas{E}$&
        $\phas{B}=\mu \phas{H}$&
        $\phas{J}_\sigma=\sigma\phas{E}$
    \end{tabular}
\end{center}
Note that $\varepsilon$,$\mu$ and $\sigma$ in this case are dependent on the position and frequency (not in time as before).
\begin{align}
    \begin{split}
        &\varepsilon=\varepsilon(\overline{r},\omega)\\[5pt]
        &\mu=\mu(\overline{r},\omega)\\[5pt]
        &\sigma=\sigma(\overline{r},\omega)
    \end{split}
\end{align}
\subsubsection*{Note on the refraction index}
The refraction index can become useful next, here we only introduce it and say what does it mean.\\
First of all we define refraction index $n$ as the square root of $\varepsilon_r$
\begin{equation}
    n=\sqrt{\varepsilon_r}=\sqrt{\frac{\varepsilon}{\varepsilon_0}}
\end{equation}
Actually this is a simplified relation, it is better to define $n$ as:
\begin{equation}
    n=\sqrt{\varepsilon_r\,\mu_r}=\sqrt{\frac{\varepsilon\,\mu}{\varepsilon_0\,\mu_0}}
\end{equation}
We already seen before that $c=\frac{1}{\sqrt{\varepsilon_0\,\mu_0}}$ and $v_p=\frac{1}{\sqrt{\varepsilon\,\mu }}$, so we the refraction index is very useful to describe the speed of an EMF through a medium:
\begin{equation}
    n=\frac{c}{v_0}
\end{equation}
\subsubsection*{Wave equation in phasor domain}
Given the system of equation in \cref{eq:maxwell_equation_in_phasor}, and applying some substitution we can obtain:
\begin{align}\label{eq:maxwell_equation_in_phasor_2}
    \begin{split}
        &\nabla\times\phas{E}=-j\omega\mu\,\phas{H}\\[5pt]
        &\nabla\times\phas{H}=j\omega \varepsilon\,\phas{E}+\sigma\phas{E}_\sigma+\overline{J}_i\\[5pt]
    \end{split}
\end{align}
It is evident that without derivative it is much more simple to find the wave equation.\\
For the sake of simplicity we will not make all the calculation to arrive to the final value, instead we use the old wave equation in time domain (\cref{eq:whave_2}) and we obtain the new wave equation in phasor domain:
\begin{equation}\label{eq:Helmholtz1}
    \nabla^2\,\phas{E}+\frac{\omega^2}{c^2}\,\phas{E}=0
\end{equation}
Most of the times this equation is written introducing the propagation constant $\gamma$ as:
\begin{equation}\label{eq:gamma_1}
    \gamma^2=-\frac{\omega^2}{c^2}=-\omega\,\mu\,\varepsilon
\end{equation}
So \cref{eq:Helmholtz1} becomes:
\begin{equation}
    \nabla^2\,\phas{E}-\gamma\,\phas{E}=0
\end{equation}
If we assume $E$ as a scalar field, then:
\begin{equation}\label{eq:Helmholtz2}
    \frac{\partial^2\phas{E}(z)}{\partial z^2}\,-\omega\,\phas{E}(z)=0
\end{equation}
\cref{eq:Helmholtz1} and \cref{eq:Helmholtz2} are also known as the Helmholtz equation (strange name, I know).
\subsubsection*{Solution of the wave equation in phasor domain}
Given \cref{eq:Helmholtz2}, it is very simple to obtain its solution:
\begin{equation}\label{eq:maxwell_solution_phasor1}
    \phas{E}=\overline{E}_{0_1}\,e^{\gamma z}+\overline{E}_{0_2}\,e^{-\gamma z}
\end{equation}
We know that $\gamma$ is a complex number because both $\mu$ and $\varepsilon$ are so:
\begin{equation}
    \gamma=\alpha+j\beta
\end{equation}
So we can write \cref{eq:maxwell_solution_phasor1} as:
\begin{equation}\label{eq:maxwell_solution_phasor}
    \phas{E}=\overline{E}_{0_1}\,e^{\alpha z}\,e^{j\beta z}+\overline{E}_{0_2}\,e^{-\alpha z}\,e^{-j\beta z}=
\end{equation}
We will focus on the forward wave equation $\phas{E}=\overline{E}_{0}\,e^{-\alpha z}\,e^{-j\beta z}$.
\subsubsection*{Going back from phasor to time domain}
If we want to go back to time domain from \cref{eq:maxwell_solution_phasor} we can just apply the relation from \cref{eq:electric_field_in_time}:
\begin{align}
    \begin{split}
        &\overline{E}(z,t)=\operatorname{Re}\left\{\phas{E}(z)e^{j\omega t}\right\}=\\[5pt]
        &=\operatorname{Re}\left\{\overline{E}_{0_2}\,e^{-\alpha z}\,e^{-j\beta z}e^{j\omega t}\right\} =\\[5pt]
        &=\operatorname{Re}\left\{|E_{0_2}|\,e^{j\varphi_0}\,e^{-\alpha z}\,e^{-j\beta z}e^{j\omega t}\right\} =\\[5pt]
        &= |E_{0_2}| e^{-\alpha z}\operatorname{Re}\left\{e^{j(\omega t-\beta z +\varphi_0)}\right\}=\\[5pt]
        &= |E_{0_2}| e^{-\alpha z}\operatorname{Re}\left\{\cos(\omega t-\beta z +\varphi_0)+j \sin(\omega t-\beta z +\varphi_0)\right\}=\\[5pt]
        &= |E_{0_2}| e^{-\alpha z}\,\cos(\omega t-\beta z +\varphi_0)
    \end{split}\label{eq:from_phas_to_time}
\end{align}
We obtained the forward equation of an EMF that propagates over the $z$ direction, that we have already seen in \cref{eq:wave_equation_generalized}.
\subsubsection*{Why $\varphi$ is a complex number?}
Consider the Maxwell equation in \cref{eq:maxwell_equation_in_phasor_2}, but without the $\overline{J}_i$ term (no current that is generating the EMF).\\
I'll not write again the \cref{eq:maxwell_equation_in_phasor_2} because you can find it simply clicking at the reference number, that being said we can do something with the second relation:
\begin{align}
    \begin{split}
        &\nabla\times\phas{H}=j\omega \varepsilon\,\left(1+\frac{\sigma}{j\omega \epsilon}\right)\phas{E}=\\[5pt]
        &\nabla\times\phas{H}=j\omega \varepsilon\,\left(1-j\frac{\sigma}{\omega \epsilon}\right)\phas{E}=\\[5pt]
        &\nabla\times\phas{H}=j\omega \underbrace{\varepsilon\,\left(1+\frac{\sigma}{j\omega \epsilon}\right)}_{\varepsilon_c}\,\phas{E}=\\[5pt]
        &\nabla\times\phas{H}=j\omega \varepsilon_c\,\phas{E}\\[5pt]
    \end{split}
\end{align}
$\varepsilon_c$ is the complex permittivity, with this the maxwell equation becomes very similar to \cref{eq:maxwell_simplified}:
\begin{equation}\label{eq:maxwell_equation_in_phasor_simplified}
    \begin{cases}
    \nabla\times\phas{E}=-j\omega\,\mu \phas{H}\\[5pt]
    \nabla\times\phas{H}=\;\;\,j\omega \, \varepsilon_c\phas{E}
    \end{cases}
\end{equation}
So \cref{eq:gamma_1} is not totally correct because we need to consider the complex permittivity $\varepsilon_c$: $\gamma^2=-\omega\,\mu\,\varepsilon_c$.\\
An interesting thing to note is that the imaginary part of $\varepsilon_c=\varepsilon\,\left(1-j\frac{\sigma}{\omega \epsilon}\right)$ are the losses during the propagation of the EMF through a medium
\begin{itemize}
    \item $\bm{\sigma}$: metallic medium loss
    \item $\bm{\omega \epsilon}$: dielectric medium loss
\end{itemize}
We can also use $\frac{\sigma}{\omega \epsilon}$ to know the property of our medium:
\begin{itemize}
    \item $\bm{\frac{\sigma}{\omega \epsilon}> 1}$: metallic medium
    \item $\bm{\frac{\sigma}{\omega \epsilon}< 1}$: dielectric medium loss
\end{itemize}
If we suppose no metallic loss: $\sigma = 0$, then:
\begin{equation*}
    \theta=0\rightarrow\varepsilon_c=\varepsilon\rightarrow\alpha=0\rightarrow\gamma =j\beta
\end{equation*}
With this simplification the forward magnetic field becomes:
\begin{equation}
    \phas{E}=\overline{E}_{0}\, e^{-j\beta z}
\end{equation}
\subsection*{EMF in frequency domain}
Until now we have seen a pure sinusoidal EMF that propagates, what if this EMF is not pure?\\
We can say that our signal is not a pure sinusoid if we have more than 1 component other than the fundamental harmonic, this mean that we deal with noisy signal.\\
Similarly to \cref{eq:electric_field_in_time}, we can define the transformation from time domain to frequency domain as:
\begin{equation}
    \overline{E}(\overline{r},t)=\frac{1}{2\pi}\int_{-\infty}^{+\infty}\overline{E}(\overline{r},w)e^{j\omega t}\,d\omega
\end{equation}
That transformation is actually the same as the one for the complex domain, but we now can consider more than 1 harmonic.\\
Again here we don't deal anymore with derivatives in time, so our job simplify a lot!:
\begin{equation}
    \frac{\partial\overline{E}(\overline{r},t)}{\partial t}=\frac{1}{2\pi}\int_{-\infty}^{+\infty}j\omega\,\overline{E}(\overline{r},w)e^{j\omega t}\,d\omega
\end{equation}
Like in \cref{eq:maxwell_equation_in_phasor_simplified} we can have a look at the Maxwell equation in phasor domain again without the $\overline{J}_i$ term (no current that is generating the EMF).\\
You can notice that they are actually the same, but in frequency the field $E$ and $H$ are dependent in time and also in frequency.
\begin{align}
    \begin{split}
        &\nabla\times\overline{E}(t,\omega)=-j\omega\mu\,\overline{H}(t,\omega)\\[5pt]
        &\nabla\times\overline{H}(t,\omega)=\;\;\,j\omega \varepsilon\,\overline{E}(t,\omega)
    \end{split}
\end{align}
And the wave equation becomes:
\begin{equation}
    \nabla^2\,\overline{E}(t,\omega)-\gamma^2\,\overline{E}(t,\omega)=0
\end{equation}
\subsection*{A little exercise}
The prof said it was little... but i'm too tired to copy all the numbers, but here I reported the passages.\\
 The request was to find the expression of the EMF ($E$ and $H$ equation) given $\gamma $, the direction $z$ and supposing no losses ($\alpha=0$). The field of $E$ is on his peak $E_0$ wen $t=0$ and $z=50$\\
 First of all the peak of the field $E_0$ is obtained when $\cos(\omega t-\beta z +\phi_0)=1$, so when $(\omega t-\beta z +\phi_0)=0$.\\
 We can simplify saying that $\omega t =0$ (we can assume the initial time $t_0=0$ because of the given data).\\
 From \cref{eq:gamma_1} we can obtain $\omega$ from $\gamma$ ($\omega^2=c^2-\omega^2$).\\
 From \cref{eq:E_with_phase_constant} we also know that $\beta=\frac{\omega}{c}$.\\
Then we obtain $\varphi_0$ from $-\frac{\omega}{c}+\phi_0=0$.\\
We have all we need to write down the equation for $E$ in time
\begin{equation*}
    E(z,t)=E_0\, \cos(-\beta z +\phi_0)\,\ihat
\end{equation*}
What about $H$?
\begin{equation}
    \nabla\times \overline{E}=\frac{\partial E_x}{\partial z}\jhat=-\mu\frac{\partial \overline{H}}{\partial t}\jhat
\end{equation}
Doing some strange calculation we can obtain the $H$ equation, where the argument of $cos$ are the same, but $H_0$ changes:
\begin{equation*}
    H(z,t)=H_0\, \cos(-\beta z +\phi_0)\,\jhat
\end{equation*}
\subsubsection*{Frequency domain}
In frequency it is much more simple:
\begin{equation*}
    \phas{E}(z)=\overline{E_0}\,\cancelto{0}{e^{-\alpha z}}\,e^{-j\beta z}=|E_0|\,e^{-j\beta z}\,e^{j\varphi_0}
\end{equation*}
We already have $E_0$, $\beta$ and $\varphi_0$ can be simply calculated as before.\\
Now $\phas{H}(z)$??
\begin{align}
    \begin{split}
        &\frac{\partial \phas{E_x}}{\partial z}=-j\omega\mu\,\phas{H}\\[5pt]
        &\phas{H}=\frac{1}{-j\omega\mu}\frac{\partial \phas{E_x}}{\partial z}=\\[5pt]
        &=\frac{j}{\omega \beta}\overline{E_0}e^{-j\beta z}=\\[5pt]
        &=\frac{\beta}{\mu \omega}|E_0|\, e^{j\varphi_0}e^{-j\beta z}
    \end{split}
\end{align}
We know that $\beta = \frac{\omega}{c}$, then we obtain the intrinsic impedance $\eta$:
\begin{equation}\label{eq:intrinsic_impedance}
    \frac{\omega}{c}\,\frac{1}{\omega \mu}=\frac{\sqrt{\mu \varepsilon}}{\mu}=\sqrt{\frac{\epsilon}{\mu}}=\eta 
\end{equation}
So we obtain a very useful equation:
\begin{equation}
    \phas{H}=\frac{1}{\eta}\,\overline{E_0}e^{-j\beta z}
\end{equation}
And we can also say that if the field propagates along $z$:
\begin{align}\label{eq:useful_maxwell_in_phasor}
    \begin{split}
        \phas{H}&=\frac{1}{\eta}\,\khat \times \phas{E}\\[5pt]
        \phas{E}&=\eta\,\phas{H}\times \khat
    \end{split}
\end{align}
Note that the second equation in \cref{eq:useful_maxwell_in_phasor} is very similar to the first hom law because we have:
\begin{equation*}
    \left[\frac{V}{m}\right]=\eta\,\left[\frac{A}{m}\right]
\end{equation*}
Just like $\left[V\right]=\Omega \,\left[A\right]$
\subsection*{One more thing: Poynting vector}
The \emph{Poynting vector} represents the directional energy flux of our radiation, this mean that it is not simply the "representation" of the power density of an EMF, but we also have the direction of the propagating wave, and it is defined by:
\begin{equation}\label{eq:poynting}
    \overline{S}=\overline{E}\times\overline{H}\rightarrow\left[\si{\frac{\watt}{\metre^2}}\right]
\end{equation}
To obtain $\overline{S}$ is not very simple, but in phasor domain it is \emph{na crema} (italian way to say "very beautiful"):
\begin{equation}
    \overline{S}=\frac{\phas{E}\times\phas{H}^*}{2}=\frac{1}{2}\,\overline{E}
_x\cdot\overline{H}_y^*\,\khat
\end{equation}