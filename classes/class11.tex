\section{Class 10 - 31/03/21}
Today we are introducing the concept of using multiple antennas to send EM signals. The main advantage of using a cluster of antennas is that we can easily change the radiation pattern without using moving parts, this will let us serve multiple directions with the same antenna geometry.
\subsection*{Two dipole array}
What does happen if we put two dipole antennas at a distance $L$ like in \cref{fig:Two_dipole_array_example}?
\begin{figure}[H]
    \begin{center}
        \begin{tikzpicture}[scale=5,tdplot_main_coords]
            \pgfmathsetmacro{\rvec}{1.2}
            \pgfmathsetmacro{\thetavec}{45}
            \pgfmathsetmacro{\phivec}{70}
            \coordinate (O) at (0,0,0);
            
            \draw[thick,->] (0,0,0) -- (1,0,0) node[anchor=north east]{$x$};
            \draw[thick,->] (0,0,0) -- (0,1,0) node[anchor=north west]{$y$};
            \draw[thick,->] (0,0,0) -- (0,0,1) node[anchor=south]{$z$};
            \draw[dashed] (0,0,0) -- (0,0,-0.3);

            \tdplotsetrotatedcoords{0}{0}{20}

            \draw[line width=0.7mm,tdplot_rotated_coords, color=blue!50!black]  (0,0.03,0.25) -- (0,0.2,0.25) node[above]{d1};
            \draw[line width=0.7mm,tdplot_rotated_coords, color=blue!50!black] (0,-0.03,0.25) -- (0,-0.2,0.25);

            \draw[line width=0.7mm,tdplot_rotated_coords, color=blue!50!black]  (0,0.03,-0.25) -- (0,0.2,-0.25) node[above]{d2};
            \draw[line width=0.7mm,tdplot_rotated_coords, color=blue!50!black] (0,-0.03,-0.25) -- (0,-0.2,-0.25);

            \draw [decorate,decoration={brace,amplitude=6pt,mirror},tdplot_rotated_coords]
            (0,-0.22,0.25) -- (0,-0.22,-0.25) node[midway,left,xshift=-1ex]{$L$};

            \tdplotsetcoord{P}{\rvec}{\thetavec}{\phivec}
            \tdplotsetcoord{P2}{\rvec +0.2}{\thetavec}{\phivec}
            \draw[dotted,tdplot_rotated_coords] (0,0,0.25) -- (P);
            \draw[dotted,tdplot_rotated_coords] (0,0,-0.25) -- (P);

            \draw[-stealth,color=red!80!black,tdplot_rotated_coords] (P) -- (P2) node[above right] {$\overline{E}(r,\varphi,\theta)$};

        \end{tikzpicture}
    \end{center} \caption{Two dipole array example}\label{fig:Two_dipole_array_example} 
\end{figure}
From \cref{eq:electric_field_antenna1}, and assuming the equivalent momentum to be $M=I_0\,d$ (true for the hertzian dipole in far field) we can write the electric field generated by a dipole:
\begin{equation}\label{eq:electric_field_antenna_hertzian}
    \overline{E}(\theta,\varphi)=j\eta\frac{I_0\,d}{2\lambda}\sin(\theta)\,\frac{e^{\,-jkr}}{r}
\end{equation}
From the example in \cref{fig:Two_dipole_array_example} we can see that the antenna are rotated by $90\si{\degree}$, so \cref{eq:electric_field_antenna_hertzian} becomes:
\begin{equation}\label{eq:electric_field_antenna_hertzian_rotated}
    \overline{E}(\theta,\varphi)=j\eta\frac{I_0\,d}{2\lambda}\cos(\theta)\,\frac{e^{\,-jkr}}{r}
\end{equation}
Then we want to find the value of the total electric field $E_T$ on far field region. The equation of $E_T$ should look like this:
\begin{align}
    \begin{split}
    \overline{E_T}(\theta,\varphi)&=\overline{E_1}(\theta,\varphi)+\overline{E_2}(\theta,\varphi)=\\[5pt]
    &=j\eta\frac{I_0\,d}{2\lambda}\left( \cos(\theta_1)\,\frac{e^{\,-jkr_1}}{r_1}\,e^{-j\frac{\delta}{2}}+ \cos(\theta_2)\,\frac{e^{\,-jkr_2}}{r_2}\,e^{j\frac{\delta}{2}} \right)
    \end{split}
\end{align}

\begin{figure}[H]
    \begin{center}
        \begin{tikzpicture}[scale=5,tdplot_main_coords]
            \pgfmathsetmacro{\rvec}{1.2}
            \pgfmathsetmacro{\thetavec}{45}
            \pgfmathsetmacro{\phivec}{70}
            \coordinate (O) at (0,0,0);
            
            \draw[thick,->] (0,0,0) -- (1,0,0) node[anchor=north east]{$x$};
            \draw[thick,->] (0,0,0) -- (0,1,0) node[anchor=north west]{$y$};
            \draw[thick,->] (0,0,0) -- (0,0,1) node[anchor=south]{$z$};
            \draw[dashed] (0,0,0) -- (0,0,-0.3);

            \tdplotsetrotatedcoords{0}{0}{20}

            \path[draw=black,fill=black] (0,0,0.25) circle (0.01cm) ;
            \path[draw=black,fill=black] (0,0,-0.25) circle (0.01cm) ;

            \tdplotsetcoord{P}{\rvec}{\thetavec}{\phivec}
            \tdplotsetcoord{P2}{\rvec +0.2}{\thetavec}{\phivec}
            \draw[dotted,tdplot_rotated_coords] (0,0,0.25) -- (P);
            \draw[dotted,tdplot_rotated_coords] (0,0,-0.25) -- (P);

            \draw[-stealth,color=red!80!black,tdplot_rotated_coords] (P) -- (P2) node[above right] {$\overline{E}(r,\varphi,\theta)$};

            \tdplotsetthetaplanecoords{\phivec-50+90}
            \tdplotdrawarc[tdplot_rotated_coords,->]{(0,0,0.35)}{0.2}{0}%
                {\thetavec+5}{anchor=south west}{$\theta_1$}
                \tdplotdrawarc[tdplot_rotated_coords,->]{(0,0,-0.45)}{0.2}{0}%
                {\thetavec-10}{anchor=south west,fill=white,text opacity=1,fill opacity=0.6}{$\theta_2$}

        \end{tikzpicture}
    \end{center} \caption{Two dipole array example}\label{fig:Two_dipole_array_example_angle} 
\end{figure}

\begin{figure}[H]
    \begin{center}
        \begin{tikzpicture}[scale=5,tdplot_main_coords]
            \pgfmathsetmacro{\rvec}{1.2}
            \pgfmathsetmacro{\thetavec}{45}
            \pgfmathsetmacro{\phivec}{70}
            \coordinate (O) at (0,0,0);
            
            \draw[thick,->] (0,0,0) -- (1,0,0) node[anchor=north east]{$x$};
            \draw[thick,->] (0,0,0) -- (0,1,0) node[anchor=north west]{$y$};
            \draw[thick,->] (0,0,0) -- (0,0,1) node[anchor=south]{$z$};
            \draw[dashed] (0,0,0) -- (0,0,-0.3);

            \tdplotsetrotatedcoords{0}{0}{20}

            \path[draw=black,fill=black] (0,0,0.25) circle (0.01cm) ;
            \path[draw=black,fill=black] (0,0,-0.25) circle (0.01cm) ;

            \tdplotsetcoord{P0}{\rvec}{\thetavec}{\phivec}
            \tdplotsetcoord{P1}{\rvec}{\thetavec}{\phivec+10}
            \tdplotsetcoord{P2}{\rvec}{\thetavec}{\phivec-10}
            %directions of E
            \draw[dotted,tdplot_rotated_coords] (0,0,-0.25) -- (0,2,0.9-0.25);
            \draw[dotted,tdplot_rotated_coords] (0,0,0.25) -- (0,2,0.9+0.25);
            \draw[dotted,tdplot_rotated_coords] (0,0,0) -- (0,2,0.9);
            %dotted little segment
            \draw[dotted,tdplot_rotated_coords] (0,0,0.25) -- (0,0.1,0.03);
            \draw[dotted,tdplot_rotated_coords] (0,0,0) -- (0,0.1,0.03-0.25);
            %
            \draw[thick,tdplot_rotated_coords,red!50!black] (0,0,0) -- (0,2/21,0.9/21) node[below,fill=white,text opacity=1,fill opacity=0.6,outer sep=0pt,inner sep=0pt,yshift=-2pt,rounded corners=2pt]{\scriptsize{$\Delta r_0$}};
            \draw[thick,tdplot_rotated_coords,red!50!black] (0,0,0.25) -- (0,2/21,0.9/21+0.25) node[below,fill=white,text opacity=1,fill opacity=0.6,outer sep=0pt,inner sep=0pt,yshift=-2pt,rounded corners=2pt]{\scriptsize{$\Delta r_1$}};
            \draw[thick,tdplot_rotated_coords,red!50!black] (0,0,-0.25) -- (0,2/21,0.9/21-0.25) node[below,fill=white,text opacity=1,fill opacity=0.6,outer sep=0pt,inner sep=0pt,yshift=-2pt,rounded corners=2pt]{\scriptsize{$\Delta r_2$}};

            
        \end{tikzpicture}
    \end{center} \caption{Two dipole array example}\label{fig:Two_dipole_array_example_far_field} 
\end{figure}